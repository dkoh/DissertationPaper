\documentclass[12pt]{report}
% Default margins are too wide all the way around. I reset them here
\setlength{\topmargin}{0in}
\setlength{\oddsidemargin}{.125in}
\setlength{\textwidth}{6.25in}
\linespread{1.5}

\usepackage{amsmath}
\usepackage{graphicx}
\usepackage{rotating}
\usepackage{verbatim}
\usepackage{url}
\usepackage{natbib}
\usepackage{setspace}
\usepackage{subfigure}
\usepackage[ruled,vlined]{algorithm2e}	
\usepackage{exscale,relsize}
\usepackage{wasysym}
\usepackage{amsfonts,amssymb, txfonts}

\begin{document}

\section{Revised Asymmetric Random Forest Variable Selection}
I am proposing 2 changes to the existing model. The initial construct of the variable selection used the following gini index

\begin{equation}\label{eqn:asymmetricginiindex}
	G(m)=\min(1-p_{left}^2-(1-p_{left})^2, 1-p_{left})
\end{equation}

The Gini here considers the purity for both classes. As we are only concern with one of the classes, I am proposing just penalizing the impurity for the class of interest. 


\begin{equation}\label{Eq:asymmetricginiindex2}
G(m)=
\begin{cases} 1-p_{left}^2-(1-p_{left})^2 & \text{if majority class in node is positive}
\\ G(m_{parent})&\text{if majority class is negative}
\end{cases}
\end{equation}

With the above equation, only when node has a majority of positive class does the GINI get calculated. Otherwise, the GINI will be set to the GINI of the node's parent.

Next, we have to adjust the method of measuring purity. Remember the split that produces the max purity will be chosen. Currently we are adopting the following

\[
	\textbf{Purity}=G(m_{parent})-pG(m_{left})-(1-p)G(m_{right})
\]
\[
	\textbf{where: } p = |m_{left}|/|m_{parent}|
\]

The distribution of $p$ is linear. A child node may have a pure class but because it only represents a small percentage of $|m|$, the resulting purity may not be significant. Thus there is a purity to node size trade-off. Because we are more interested in purity than the size of the node, we would like to put less weight on the node size. Thus we propose the following

\[
	\textbf{Purity}=G(m_{parent})-pG(m_{left})-(1-p)G(m_{right})\\
\]
\[
\textbf{where: } p = \frac{log(|m_{left}|/|m_{parent}|)}{log(|m_{left}|/|m_{parent}|) +log(|m_{right}|/|m_{parent}|)}
\]
The formulation reduces the penalization for a small number of observations in the child node relative to its parent.

Running such a formulation has yield success in terms of finding pertinent variables based on simulated data. In addition, it has also been effective in finding asummetric effects when there is interaction.

\end{document}  